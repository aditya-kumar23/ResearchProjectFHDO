\chapter{Introduction}\label{chap:introduction}

\section{Motivation} \label{sec:motivation}
In recent years, a sharp increase in the use of autonomous systems in a multitude of domains has driven significant advances in the field of robotics. These systems have been used in various industries, such as healthcare \cite{silvera2024robotics}, transportation \cite{sevgi2024using}, and manufacturing \cite{kaur2025robotics}, to improve efficiency, precision, and overall performance.

In this increasingly interconnected and complex world, the limitations of single-agent, centralised, monolithic systems are becoming more apparent. As a result, there is a growing trend towards the development of multi-agent systems that can work collaboratively and adaptively to tackle complex problems. These systems are able to distribute decision-making and problem-solving processes across multiple agents, allowing greater flexibility, scalability, and efficiency in the face of dynamic environments. A multi-agent system is selected as a solution to problems that are naturally more distributed, large, and complex for a single agent to solve. Implementation of these systems in problem domains such as search and rescue operations \cite{delmerico2019current}, environmental monitoring \cite{grocholsky2006cooperative}, precision agriculture \cite{ribeiro2021multi} where these systems can be used to monitor crop health and control pest infestations, and mapping of archaeological sites \cite{patel2021collaborative}. 

While multi-agent systems are distributed in their true nature, the design and architecture for coordination and communication between multiple agents must be carefully planned to ensure seamless collaboration and task completion. This involves developing algorithms that enable agents to share information, make decisions collectively, and adapt to changing environments in real time. 
A fundamental requirement for these coordinated tasks is a shared, real-time understanding of the environment, often represented as a global map. This is especially critical in scenarios where external positioning systems like GPS are inconsistent or unavailable entirely, necessitating a self-contained perception solution. For a traditional single-agent system this is solved by Simultaneous Localisation and Mapping  \emph{(SLAM)} \cite{taheri2021slam}, which allows the agent to build a map of its environment while simultaneously localising itself within that map. In a multi-agent setting this is extended to collaborative SLAM.  \emph{C-SLAM} leverages multiple robots working together to overcome the limitations of single-agent systems, such as limited perspectives, error accumulation over time, and challenges of large-scale complex environments. By sharing sensory information and coordinating efforts, multi-agent teams can create more accurate and robust maps while also increasing operational efficiency \cite{zou2019collaborative}.


\section{Problem statement} \label{sec:problemstate}
While many factors contribute to successful development and deployment of a C-SLAM system, architecture plays a vital role. Architectural choices determinehow sensing, communication, and optimisation workloads are distributed across robots and infrastructure, and directly affect scalability, robustness, and real-time performance. Key architectural paradigms, ranging from \emph{purely centralised} setup where all map building and optimisation tasks are offloaded on a centralised server to \emph{decentralised} where each robot maintains a local estimate and global consistency is recovered through distributed coordination. Each paradigm introduces inherent trade-offs in estimation accuracy, communication bandwidth requirements, computational load, and tolerance to network impairments.

Architectural selection in C-SLAM systems often relies on intuition, with decentralized architectures recognized for their robustness and scalability due to eliminating single points of failure. However, they may incur higher latency from increased coordination needs. In contrast, centralized pipelines offer faster global consistency but can become bottlenecks as team sizes grow or network conditions worsen. Despite these widely held assumptions and a substantial existing research on C-SLAM architectures and optimisation techniques, most studies evaluate systems under idealised or loosely controlled communication conditions, or focus on a single architectural paradigm in isolation. As a result, there is a lack of systematic, controlled and reproducible comparative assessment that isolates architectural effects, characterises latency and convergence under increasing collaboration scale, and relates performance to computational and communication constraints. Addressing this gap is necessary to enable principled architectural selection rather than assumption-driven design.

\section{Research goals}\label{sec:goals}

The primary objective of this research project is to conduct and establish an evidence-based understanding of how architectural choices in C-SLAM shape the core trade-offs between \emph{global consistency}, \emph{robot autonomy}, and system performance. To support principled architecture selection, this work aims to quantify and relate key charactersitcs such as latency, scalability limits, robustness and implementation complexity. The key research questions to be addressed are:

\begin{enumerate}
    \item How do architectures differ in terms of global map update latency and convergence time? 
    \item What is the impact on latency as the number of collaborating agents increases?
    \item What are the computational loads and communication bandwidth requirements of each architecture under varying network constraints, and how do these evolve under constrained network conditions?
    \item How does the choice of architecture affect the system's robustness to common real-world challenges?
\end{enumerate}

\section{Structure of the Thesis}
This thesis is organised into six chapters. Chapter 2 introduces the fundamental concepts of graph based SLAM and C-SLAM, establishing the theoretical basis for the subsequent analysis. Chapter 3 reviews the state of the art in collaborative SLAM and builds a comparative analysis grounded in existing literature. Chapter 4 outlines the structure of the comparative framework and the design of the simulation-based evaluation strategy. Chapter 5 details the implementation of the experimental testbed, including the deployment of representative collaborative SLAM architectures. Chapter 6 presents and analyses the comparative results with respect to performance, latency, resource utilisation, and robustness. Finally, Chapter 7 summarises the key findings, draws conclusions, and discusses directions for future research.

